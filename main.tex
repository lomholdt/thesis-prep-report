\documentclass{article}
\usepackage[utf8]{inputenc}
\usepackage[]{hyperref}
\usepackage{todonotes}

\usepackage{apacite}

\title{Thesis Preparation Report}
\author{Jonas Tonny Nielsen \and Jonas Lomholdt}
\date{\today}

\begin{document}

\maketitle

\listoftodos

\tableofcontents
%\listoffigures
%\listoftables

\todo{handin Jan 8th}

\section{Introduction} % Introduction to the project
A number of studies have shown that introducing clicker systems into classrooms has a positive effect on learning outcomes and classroom interactivity \cite{yourstone2008classroom, siau2006use, lantz2014effectiveness}. 



%\cite{yourstone2008classroom}
%\maskciteyear{einstein}. 

\section{Literature Review} % What is already known and done + existing solutions?
Many different classroom response system solutions already exists, and all seem to have a different approach. Some are made exclusively for children or ground-schoolers (\url{https://kahoot.it}), and some are feature rich, made for conferences and classrooms (\url{http://socrative.com},  \url{https://www.polleverywhere.com} and \url{https://www.iclicker.com/} etc.). The existing literature is centered around physical clickers and the learning outcomes of using them, though there seems to be a lack of research of the use of such systems within the last few years, where smartphones has become widespread/popular among students \cite[p.~1]{stowell2015use}. The least resent literature dating back to around 2007, is mainly focused on physical clicker devices, where the students buy actual "remote control"-looking devices, where more modern solutions (including the smartphone) has not yet matured.




%%% Talk about \cite{stowell2015use} and his comparision between physical clickers and mobile

One of the main reasons to use a classroom response systems is the importance of interactivity in learning \cite{draper2004increasing}.  Studies find that classroom response systems engage interactivity \todo{(ref)} and helps students stay active during lectures. \cite[p.~116]{moredich2007engaging}.

\section{Method}


\section{Implementation - Our Contribution - how we are improving this}
Our idea is to implement a classroom response system from scratch. The system should include features that support asking and answering technical questions.

\subsection{Possible implementation thoughts}

\subsection{Features} % The features of the system

%Users
    %user roles
        %Admin
        %Teacher
        %Student
    %Profile picture
%Classrooms
    %Search for classrooms
    %Create classroom (teachers only)
    %Invite students to classrooms
    %Group questions by tags
    %Invite to classroom by email
    %Request access to classroom
    %Delete classrooms
%Questions
    %Create questions of different types (teachers only)
        %Multiple Choice questions
        %Include pictures
        %Fill-in-the-blank questions
    %Questions should be open and closeable by teachers (students can/cannot answer)
    %Students should be notified when new questions are available
    %Questions should be publishable
    %Delete questions
    %Add tags/labels
%Dashboard (teacher)
    %Graphs with answer overview
    %Real-time feed of answers (Pusher/Sockets, green check mark if answered red cross if not-ish)
%Dashboard (student)
    %See classrooms
    %See question history


% RESTfull service 
% thesis.com/classroom/advanced-programming-2015/question/16/

\subsection{Database design}


\missingfigure{Database Design}






%--------------------REFERENCES--------------------%
\bibliographystyle{apacite}
\bibliography{ref.bib}


\end{document}
